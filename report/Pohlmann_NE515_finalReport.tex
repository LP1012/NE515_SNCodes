\documentclass{NE515}
\usepackage{physics}
\usepackage{booktabs}

\onehalfspacing
\theoremstyle{definition}
\newtheorem{definition}{Definition}[section]

\title{\textbf{Final Report: 1D Discrete Ordinates Code}\\ \vspace{1em}
NE515: Radiation Interaction and Transport}
\author[1]{Liam Pohlmann}
%\author[2]{Someone else\ldots}
\affil[1]{University of New Mexico, Department of Nuclear Engineering}
%\affil[2]{University of New Mexico, Department of \ldots}

\date{\textbf{Due: 11 December 2024}}

\begin{document}
    \maketitle
    \tableofcontents
    \listoffigures
    \clearpage

    \section*{Preface}
    The code for this project was developed and uploaded to a GitHub repository for improved code readability and version control.
    The hope is that the results of this work may go on to help other students in the industry who are also getting started learning about radiation transport.


    \section{Premise and Theory}
    The purpose of this report is to outline the steps taken and results generated for the solution of the linear Boltzmann transport equation on a 1D slab subject to varying conditions.
    In the application of radiation transport, the general 1D, timme-independent, monoenergetic transport equation can be written as:
    \begin{equation}
        \label{eq:1d-transport-equation}
        \mu \pdv[]{\psi }{z}\left( z,\mu ) \right)+\Sigma_t \psi (z,\mu)=\left( \frac{\Sigma_s+\nu\Sigma_f}{4\pi} \right)\phi(z)+S(z)
    \end{equation}
    were $\mu=\cos(\vec{\Omega}'\cdot \vec{\Omega})$, the cosine of the scattering angle, $Sigma_t$ is the total macroscopic cross-section, $\Sigma_s$ is the zero\textsuperscript{th}-order macroscopic scattering cross-section, $\psi(z,\mu)$ is the angular neutron flux, and $S(z)$ is a generalized, isotropic volumetric source.
    Additionally, the variable of interest, $\phi(z)$, is the scalar neutron flux, defined as:
    \begin{equation}
        \label{eq:phi-definition}
        \phi(z)=\int_{4\pi}^{}\psi(z,\vec{\Omega})\dd \vec{\Omega}.
    \end{equation}
    The above is subject to boundary conditions on the edges of the slab, namely $\psi(z=0,\mu)$ and $\psi(z=L,\mu)$ for a slab of length $L$.

    \Cref{eq:1d-transport-equation} is known as an \textit{integro-differential equation}, as it is an equation that contains a function, its derivative, and its integral.
    To solve this, one must discretize both the derivatives and integrals in a set of algebraic relations, then solve.

    First, the domain is discretized uniformly using a finite volume approach in the spacial axis.
    That is, each volume (in 1D geometry, this is just a line segment) holds the average function value as numerically computed at its \textit{center}.
    \begin{equation}
        \bar{\psi}_{n,i}=\frac{1}{\Delta z}\int_{z_{i-\nicefrac{1}{2}}}^{z_{i+\nicefrac{1}{2}}} \psi_n(z)\dd z
    \end{equation}
    where $\bar{\psi}_{n,i}$ is the \textit{cell-averaged} angular flux (and will be henceforth written as $\psi_{n,i}$) at $z=\Delta z\left( i+\nicefrac{1}{2}\right)$ and $\mu=\mu_n\in[\mu_1,\mu_2,\ldots,\mu_N]$, $\Delta z$ is the cell width, and $i\in[1,2,\ldots,I]$.
    This implies that functional relations must be made not only for the function values themselves, but also of the numerical ``fluxes'' at the boundaries of each cell.

    Though discretization of the slab spatially is straightforward, the choice of discrete angular values is an opportunity to improve numerical accuracy while also reducing computational effort.
    Gaussian quadrature is a technique using in numerical integration in which the error is minimized by selecting particular weights and multiplying them by function values evaluated at specially-selected points.
    The generalized case can be written as~\cite{gezerlisNumericalMethodsPhysicsa}:
    \begin{equation}
        \int_{-1}^{1} f(x) \dd x = \sum_{k=1}^{K-1} w_k f(x_k)
    \end{equation}
    The values of the $x_k$'s (the ``nodes'') turn out to be zeros of Legendre polynomials of order $K$ on the standard interval $[-1,1]$.
    Legendre polynomials are a family of equations that share a common set of favorable properties, such as orthogonality, and can be generated using Rodrigues' formula~\cite{LegendrePolynomials2024}:
    \begin{equation}
        \label{rodrigues forumula}
        P_K (x)= \frac{1}{2^K} \sum_{k=0}^{K}
        \begin{pmatrix}
            n \\
            k
        \end{pmatrix}^2
        \left( \frac{x-1}{2} \right)^k.
    \end{equation}
    The weights for Gauss quadrature (the $c_k$'s) can be calculated with~\cite{gezerlisNumericalMethodsPhysicsa}:
    \begin{equation}
        \label{gauss weights}
        c_k=\frac{2}{\left( 1-x_k^2 \right)\left[ P' (x_k) \right]^2}
    \end{equation}
    Instead of directly applying\cref{rodrigues forumula,gauss weights}, a Gaussian quadrature library for C++ was directly implemented for numerical integration.

    By assuming azimuthal independence, we rewrite \cref{eq:phi-definition} as:
    \begin{equation}
        \phi(z)=\int_{4\pi}^{}\psi(z,\vec{\Omega})\dd \vec{\Omega} = \int_{0}^{2\pi } \dd \Omega \int_{-1}^{1} \psi(z,\mu)\dd \mu.
    \end{equation}
    Using the previous notation, \cref{eq:1d-transport-equation} can be written in terms of discrete scattering angles, or \textit{discrete ordinates}, as:
    \begin{equation}
        \mu_n \pdv[]{\psi_n (z)}{z}+\Sigma_t(z)\psi_n(z)=\frac{\Sigma_s}{2}\sum_{k=1}^{N} w_k \psi_k(z)+S(z)
    \end{equation}
    Letting:
    \begin{equation}
        q^{(j-1)}(z) = \frac{\Sigma_s}{2}\sum_{k=1}^{N} w_k \psi^(j)_k(z)+S(z)
    \end{equation}
    and:
    \begin{equation}
        q_{n,i}=\bar{q(z)}=\frac{1}{\Delta z_i}\int_{z_{i-\nicefrac{1}{2}}}^{z_{i+\nicefrac{1}{2}}} q_n(z)\dd z
    \end{equation}
    a source iteration is created with:
    \begin{equation}
        \label{source iteration}
        \mu_n \dv[]{\psi_n^(j)(z)}{z}+\Sigma_t(z)\psi_n^(j)(z)=q^{(j-1)}(z)
    \end{equation}
    By integrating \cref{source iteration} over an arbitrary cell, we find:
    \begin{equation}
        \mu_n\left( \psi_{n,i+\nicefrac{1}{2}}^(j)-\psi_{n,i-\nicefrac{1}{2}^(j)} \right)+\Sigma_{t,i}\Delta z_i \psi_{n,i}^(j)=\Delta z_i q_{n,i}^{((j)-1)}
    \end{equation}

    To iterate the source, first the solution is swept in one direction along $z$ with the flow of neutrons, then is swept in the opposite direction to solve for the neutrons flowing the other way.
    We define the following differencing closure:
    \begin{subequations}
        \label{differencing closure}
        \begin{equation}
            \psi_{n,i}=(1-\alpha)\psi_{n,i+\nicefrac{1}{2}}+\alpha \psi_{n,i-\nicefrac{1}{2}},\quad \mu_n>0
        \end{equation}
        \begin{equation}
            \psi_{n,i}=\alpha\psi_{n,i+\nicefrac{1}{2}}+(1-\alpha) \psi_{n,i-\nicefrac{1}{2}},\quad \mu_n<0
        \end{equation}
    \end{subequations}
    corresponding to forward and backward particle flow, respectively.
    When $\alpha=\nicefrac{1}{2}$, \cref{differencing closure} is known as \textit{diamond differencing}, while when $\alpha=0$, \cref{differencing closure} is known as \textit{step differencing}.
    These correspond to quadratic and linear convergence, respectively.
    Thus, the following two algorithms were implemented:
    \begin{subequations}
        \begin{equation}
            \label{forward sweep}
            \psi_{n,i}^(j)=\left( \frac{1}{1+\frac{\Sigma_{t,i}\Delta z_i}{2\mu_n}} \right)\left( \psi_{n,i-\nicefrac{1}{2}}+\frac{\Delta z_i q_{n,i}^{(j-1)}}{2\mu_n} \right), \quad n=1,2,\ldots,\frac{N}{2}
        \end{equation}
        \begin{equation}
            \label{backward sweep}
            \psi_{n,i}^(j)=\left( \frac{1}{1+\frac{\Sigma_{t,i}\Delta z_i}{2|\mu_n|}} \right)\left( \psi_{n,i+\nicefrac{1}{2}}+\frac{\Delta z_i q_{n,i}^{(j-1)}}{2|\mu_n|} \right), \quad n=\frac{N}{2}+1,\frac{N}{2}+2,\ldots,{N}
        \end{equation}
    \end{subequations}
    along with:
    \begin{equation}
        \label{flux relation}
        \psi_{n,i+\nicefrac{1}{2}}=2\psi_{n,i}-\psi_{n,i-\nicefrac{1}{2}}
    \end{equation}

    \subsection{Numerical Considerations}
    Since negative flux is non-physical but numerically possible, particularly if the mesh is not fine enough for larger $\alpha$.
    When such an event occurs, the code was implemented with a ``negative flux fixup'', which changes from diamond differencing to step differencing when the angular flux dips below zero.

    In addition, all boundary source terms were normalized to 1 for convenience.
    If $\Gamma_{n,\nicefrac{1}{2}}=\Gamma_{n,L}$ and $\Gamma_{n,I+\nicefrac{1}{2}}=\Gamma_{n,L}$ are boundary flux terms, the following algorthm was implemented:
    \begin{equation}
        \label{numerical flux scaling}
        \int_{4\pi }^{} \Gamma_{L/R,n}\dd \vec{\Omega} \approx 2\pi \sum_{n=0}^{N/2} w_n\mu_n \Gamma_{L/R,n}\equiv \gamma_n
    \end{equation}
    Then: $\Gamma_{L/R,n}\rightarrow \nicefrac{\Gamma_{L/R,n}}{\gamma_N}$.


    \section{Code Verification: Method of Manufactured Solutions}
    To verify the above numerical, the Method of Manufacture Solutions was implemented.
    That is, a solution was specified of the form:
    \begin{equation}
        \psi^m (z,\mu) = C f(z) g(\mu)
    \end{equation}
    and substituted into \cref{eq:1d-transport-equation}.
    Assuming no fission and solving for the manufactured volumetric source:
    \begin{equation}
        \label{man source eq}
        Q^m(z,\mu) = \mu \pdv[]{\psi }{z}+\Sigma_t\psi(z,\mu)-\frac{\Sigma_s}{4\pi}\phi(z)
    \end{equation}
    By choosing a shifted Legendre polynomial that is \textit{strictly positive} in $\mu$ and a simple quadratic profile in $z$ with zeros on the boundaries $[0,L]$, the following function was selected:
    \begin{equation}
        \label{manufactured solution}
        \begin{split}
            \psi_{m}(z,\mu)&=\frac{1}{\pi L^2}z(L-z)\left[ P_3 (\mu)+1\right]\\
            &= \frac{1}{2\pi L^2}z(L-z)\left[ 5\mu^3-3\mu+2\right]
        \end{split}
    \end{equation}
    Substituting \cref{manufactured solution} into \cref{man source eq}, we get a manufactured source in terms of $P_3(\mu)$ as:
    \begin{equation}
        \begin{split}
            Q^m(z,\mu)&=\frac{\mu }{\pi L^2}\left[ P_3(\mu)+1 \right]\left( L-2z \right) +\\
            &+\frac{\Sigma_t}{\pi L^2}z\left( L-z \right)\left[ P_3(\mu)+1 \right]+\\
            &-\frac{\Sigma_s}{\pi L^2}z\left( L-z \right)
        \end{split}
    \end{equation}
    When \cref{manufactured solution} is integrated over the solid angle, the analytical scalar is given as:
    \begin{equation}
        \phi^m(z)=\frac{4}{L^2}z\left( L-z \right)
    \end{equation}

    Similar to the previous section, we write a discrete-ordinates transport form of \cref{man source eq} as:
    \begin{equation}
        \mu_n \dv[]{\psi_{n}^m}{z}+\Sigma_t \psi_n ^m (z)=\frac{\Sigma_s}{4\pi}\phi^m(z) + S_{n,i}
    \end{equation}
    where we define:
    \begin{equation}
        \label{manufactured source}
        S_{n,i}=\frac{1}{\Delta _z}\int_{z_{i-\nicefrac{1}{2}}}^{z_{i+\nicefrac{1}{2}}}Q^m(\mu_n,z)\dd z.
    \end{equation}
    \Cref{manufactured source} requires the use of numerical integration again, and to maintain the quadratic convergence from the diamond differencing, a second-order numerical quadrature scheme must also be implemented.
    One of the simplest choices is the Trapezoid Rule~\cite{gezerlisNumericalMethodsPhysicsa}, which when applied to \cref{manufactured source} gives:
    \begin{equation}
        S_{n,i}\approx\frac{1}{2}\left( Q^m(\mu_n,z_{i+\nicefrac{1}{2}})+Q^m(\mu_n,z_{i-\nicefrac{1}{2}} \right)
    \end{equation}

    When implemented over various number of cells, the numerical solution converged near-quadratically to the manufactured solution.
    A plot of the numerical solution can be found in \cref{true-solution-convergence}.
    A convergence study was also conducted and can be found in \cref{convergence-study}.


    \begin{figure}[!htb]
        \centering
        \includegraphics[width=0.8\textwidth]{../SN1_validation/plots/MMS_plot}
        \caption{Numerical solution plotted against analytical manufactured solution.}
        \label{true-solution-convergence}
    \end{figure}

    \begin{figure}[!htb]
        \centering
        \includegraphics[width=0.8\textwidth]{../SN1_validation/plots/L2error_plot}
        \caption{Convergence study of MMS solution using the discrete L2-norm.}
        \label{convergence-study}
    \end{figure}


    \section{Single-Section 1D Slab}
    The following cases were run with the given parameters and boundary conditions.
    Cases 1--5 had their conditions hard-coded in because they were based on the code titled \texttt{SN1}, while case 5 was based off of \texttt{SN2}, which allowed for a variable input using a text file.
    For all cases in this section, the following parameters were implemented:
    \begin{itemize}
        \item Isotropic scattering, $\Sigma_{s,l}=0$ for $l\geq 1$.
        \item Gaussian quadrature order, $N=8$.
        \item Total cross-section, $\Sigma_t=1.0$.
        \item No fission, $\Sigma_f=0.0$.
        \item Source iteration tolerance, $\epsilon=10^{-6}$.
    \end{itemize}

    \subsection{Cases 1 \& 2}
    \begin{itemize}
        \item Incident beam on left boundary moving along the most forward direction.
        \item Vacuum boundary on right boundary.
        \item No volumetric source.
        \item $\Sigma_s=0.5$.
        \item $L=5.0$.
        \item $I=[10,20,50,100]$.
        \item $\alpha=0.5$ (Case 1) and $\alpha=0.0$ (Case 2).
    \end{itemize}
    \begin{figure}
        \centering
        \begin{subfigure}{0.45\linewidth}
            \centering
            \includegraphics[width=\linewidth]{../SN_Cases/plots/10_case1_out_plot}
            \caption{$I=10$ Cells}
        \end{subfigure}
        \hfill
        \begin{subfigure}{0.45\linewidth}
            \centering
            \includegraphics[width=\linewidth]{../SN_Cases/plots/20_case1_out_plot}
            \caption{$I=20$ Cells}
        \end{subfigure}
        \hfill
        \begin{subfigure}{0.45\linewidth}
            \centering
            \includegraphics[width=\linewidth]{../SN_Cases/plots/50_case1_out_plot}
            \caption{$I=50$ Cells}
        \end{subfigure}
        \hfill
        \begin{subfigure}{0.45\linewidth}
            \centering
            \includegraphics[width=\linewidth]{../SN_Cases/plots/100_case1_out_plot}
            \caption{$I=100$ Cells}
        \end{subfigure}

        \caption{Case 1 generated plots.}
        \label{fig:case-1}
    \end{figure}
    \clearpage

    \begin{figure}
        \centering
        \begin{subfigure}{0.45\linewidth}
            \centering
            \includegraphics[width=\linewidth]{../SN_Cases/plots/10_case2_out_plot}
            \caption{$I=10$ Cells}
        \end{subfigure}
        \hfill
        \begin{subfigure}{0.45\linewidth}
            \centering
            \includegraphics[width=\linewidth]{../SN_Cases/plots/20_case2_out_plot}
            \caption{$I=20$ Cells}
        \end{subfigure}
        \hfill
        \begin{subfigure}{0.45\linewidth}
            \centering
            \includegraphics[width=\linewidth]{../SN_Cases/plots/50_case2_out_plot}
            \caption{$I=50$ Cells}
        \end{subfigure}
        \hfill
        \begin{subfigure}{0.45\linewidth}
            \centering
            \includegraphics[width=\linewidth]{../SN_Cases/plots/100_case2_out_plot}
            \caption{$I=100$ Cells}
        \end{subfigure}

        \caption{Case 2 generated plots.}
        \label{fig:case-2}
    \end{figure}

    \subsection{Case 3}
    \begin{itemize}
        \item \textit{Isotropic} incident beam on left boundary, normalized to unit incidence current.
        \item Vacuum boundary on right boundary.
        \item No volumetric source.
        \item $\Sigma_s=0.5$.
        \item $L=5.0$.
        \item $I=[10,20,50,100]$.
        \item $\alpha=0.0, 0.5$.
    \end{itemize}

    \begin{figure}
        \centering
        \begin{subfigure}{0.45\linewidth}
            \centering
            \includegraphics[width=\linewidth]{../SN_Cases/plots/10_case3_step_out_plot}
            \caption{$I=10$ Cells}
        \end{subfigure}
        \hfill
        \begin{subfigure}{0.45\linewidth}
            \centering
            \includegraphics[width=\linewidth]{../SN_Cases/plots/20_case3_step_out_plot}
            \caption{$I=20$ Cells}
        \end{subfigure}
        \hfill
        \begin{subfigure}{0.45\linewidth}
            \centering
            \includegraphics[width=\linewidth]{../SN_Cases/plots/50_case3_step_out_plot}
            \caption{$I=50$ Cells}
        \end{subfigure}
        \hfill
        \begin{subfigure}{0.45\linewidth}
            \centering
            \includegraphics[width=\linewidth]{../SN_Cases/plots/100_case3_step_out_plot}
            \caption{$I=100$ Cells}
        \end{subfigure}

        \caption{Case 3 generated plots with $\alpha = 0.0$ (step differencing).}
        \label{fig:case-3-step}
    \end{figure}


    \begin{figure}
        \centering
        \begin{subfigure}{0.45\linewidth}
            \centering
            \includegraphics[width=\linewidth]{../SN_Cases/plots/10_case3_diamond_out_plot}
            \caption{$I=10$ Cells}
        \end{subfigure}
        \hfill
        \begin{subfigure}{0.45\linewidth}
            \centering
            \includegraphics[width=\linewidth]{../SN_Cases/plots/20_case3_diamond_out_plot}
            \caption{$I=20$ Cells}
        \end{subfigure}
        \hfill
        \begin{subfigure}{0.45\linewidth}
            \centering
            \includegraphics[width=\linewidth]{../SN_Cases/plots/50_case3_diamond_out_plot}
            \caption{$I=50$ Cells}
        \end{subfigure}
        \hfill
        \begin{subfigure}{0.45\linewidth}
            \centering
            \includegraphics[width=\linewidth]{../SN_Cases/plots/100_case3_diamond_out_plot}
            \caption{$I=100$ Cells}
        \end{subfigure}

        \caption{Case 3 generated plots with $\alpha = 0.5$ (step differencing).}
        \label{fig:case-3-diamond}
    \end{figure}

    \subsection{Case 4}
    \begin{itemize}
        \item Incident beam on left boundary moving along the most forward direction.
        \item Vacuum boundary on right boundary.
        \item No volumetric source.
        \item $\Sigma_s=0.99$.
        \item $L=100.0$.
        \item $I=100$.
        \item $\alpha=0.5$.
    \end{itemize}

    \begin{figure}[!htb]
        \centering
        \includegraphics[width=0.8\textwidth]{../SN_Cases/plots/100_case4_out_plot}
        \caption{Case 4 Generated Plot}
        \label{fig:case-4}
    \end{figure}

    \subsection{Case 5}
    \begin{itemize}
        \item Incident beam on left boundary.
        \item Albedo (reflected) condition on right boundary.
        \item $\Sigma_s=1.0$
        \item $L=5.0$
        \item $I=20$
        \item Albedo coefficient, $\gamma=0.5,1.0$
    \end{itemize}

    \begin{figure}
        \centering
        \begin{subfigure}{0.45\linewidth}
            \centering
            \includegraphics[width=\linewidth]{../SN_Cases/plots/20_case5_albedo-0.5_out_plot}
            \caption{$I=20$ Cells, $\gamma = 0.5$}
        \end{subfigure}
        \hfill
        \begin{subfigure}{0.45\linewidth}
            \centering
            \includegraphics[width=\linewidth]{../SN_Cases/plots/20_case5_albedo-1.0_out_plot}
            \caption{$I=20$ Cells, $\gamma = 1.0$}
        \end{subfigure}
        \hfill
        \begin{subfigure}{0.45\linewidth}
            \centering
            \includegraphics[width=\linewidth]{../SN_Cases/plots/100_case5_albedo-0.5_out_plot}
            \caption{$I=100$ Cells, $\gamma = 0.5$}
        \end{subfigure}
        \hfill
        \begin{subfigure}{0.45\linewidth}
            \centering
            \includegraphics[width=\linewidth]{../SN_Cases/plots/100_case5_albedo-1.0_out_plot}
            \caption{$I=20$ Cells, $\gamma = 1.0$}
        \end{subfigure}

        \caption{Case 5 generated plots with both albedo conditions: $\gamma = 0.5, 1.0$.}
        \label{fig:case-5}
    \end{figure}

    \subsection{Case 6}
    \begin{itemize}
        \item Spatially uniform isotropic source, normalized such that the total number of neutrons in the slab is unity.
        \item Vacuum boundary conditions
        \item $\Sigma_s=0.9,0.99$
        \item $L=100$
        \item $I=100$
        \item $\alpha=0.0,0.5$
    \end{itemize}

    \begin{figure}
        \centering
        \begin{subfigure}{0.45\linewidth}
            \centering
            \includegraphics[width=\linewidth]{../SN2/plots/freeSurface1_alpha0_input_out.csv}
            \caption{$\Sigma_s=0.9$, $\alpha=0.0$}
        \end{subfigure}
        \hfill
        \begin{subfigure}{0.45\linewidth}
            \centering
            \includegraphics[width=\linewidth]{../SN2/plots/freeSurface1_diamond_input_out.csv}
            \caption{$\Sigma_s=0.9$, $\alpha=0.5$}
        \end{subfigure}
        \hfill
        \begin{subfigure}{0.45\linewidth}
            \centering
            \includegraphics[width=\linewidth]{../SN2/plots/freeSurface2_alpha0_input_out.csv}
            \caption{$\Sigma_s=0.99$, $\alpha=0.0$}
        \end{subfigure}
        \hfill
        \begin{subfigure}{0.45\linewidth}
            \centering
            \includegraphics[width=\linewidth]{../SN2/plots/freeSurface2_diamond_input_out.csv}
            \caption{$\Sigma_s=0.99$, $\alpha=0.5$}
        \end{subfigure}

        \caption{Case 6 generated plots with both scattering conditions, $\Sigma_s = 0.9, 0.99$, and both step and diamond differencing ($\alpha=0.0,0.5$).}
        \label{fig:case-6}
    \end{figure}


    \section{Reed Problem}
    As a final, and arguably most important aside from the MMS solution, demonstration of the written code, the so-called Reed problem was implemented.
    First, the basic code \texttt{SN1} was generalized to allow for arbitrary number of regions in a slab, with each region being able to handle different physical parameters and cell sizes.
    This application is meant to test the codes ability to handle sudden, drastic changes in material properties.
    The parameters for this can be found in \cref{tab:reed-params}.
    \begin{table}
        \centering
        \begin{tabular}{lrrrrr}
            \toprule
            \textbf{Region No.}   & \textbf{1} & \textbf{2} & \textbf{3} & \textbf{4} & \textbf{5} \\
            \midrule
            $\Sigma_t$            & 50.0       & 5.0        & 0.0        & 1.0        & 1.0        \\
            $\Sigma_{s}/\Sigma_t$ & 0.0        & 0.0        & 0.0        & 0.9        & 0.9        \\
            Source Strength       & 50.0       & 0.0        & 0.0        & 1.0        & 0.0        \\
            Length                & 2.0        & 1.0        & 2.0        & 1.0        & 2.0        \\
            \bottomrule
        \end{tabular}
        \caption{Reed Problem Parameters}
        \label{tab:reed-params}
    \end{table}

    To improve numerical accuracy, $N=32$ quadrature nodes were implemented, along with a total of 900 cells.
    The input file which was used to generate the numerical results can be found in \cref{sec:reed-input-file}.
    As can be seen in \cref{fig:reed-figure}, the code performed very well, matching the expected result provided.


    \begin{figure}[!htb]
        \centering
        \includegraphics[width=0.9\textwidth]{../SN2/plots/reed_out.csv}
        \caption{Generated plot from Reed problem application.}
        \label{fig:reed-figure}
    \end{figure}
    
    \appendix
    \section{Reed Input File}\label{sec:reed-input-file}
    \inputminted[linenos, bgcolor=LightGray, fontsize=\footnotesize]{text}{../SN2/reed_input.txt}
    \bibliography{NE515_final}
    \bibliographystyle{ieeetr}


\end{document}
